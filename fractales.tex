%%
% Modificación de una plantilla de Latex para adaptarla al castellano.
%%

%%%%%%%%%%%%%%%%%%%%%
% Thin Sectioned Essay
% LaTeX Template
% Version 1.0 (3/8/13)
%
% This template has been downloaded from:
% http://www.LaTeXTemplates.com
%
% Original Author:
% Nicolas Diaz (nsdiaz@uc.cl) with extensive modifications by:
% Vel (vel@latextemplates.com)
%
% License:
% CC BY-NC-SA 3.0 (http://creativecommons.org/licenses/by-nc-sa/3.0/)
%
%%%%%%%%%%%%%%%%%%%%%

%----------------------------------------------------------------------------------------
%	PACKAGES AND OTHER DOCUMENT CONFIGURATIONS
%----------------------------------------------------------------------------------------

\documentclass[a4paper, 11pt]{article} % Font size (can be 10pt, 11pt or 12pt) and paper size (remove a4paper for US letter paper)

\usepackage[protrusion=true,expansion=true]{microtype} % Better typography
\usepackage{graphicx} % Required for including pictures
\usepackage[usenames,dvipsnames]{color} % Coloring code
\usepackage{wrapfig} % Allows in-line images
\usepackage[utf8]{inputenc}

% Imágenes
\usepackage{graphicx} 

\usepackage{amsmath}

% sudo apt-get install texlive-lang-spanish
\usepackage[spanish]{babel} % English language/hyphenation
\selectlanguage{spanish}
% Hay que pelearse con babel-spanish para el alineamiento del punto decimal
\decimalpoint
\usepackage{dcolumn}
\newcolumntype{d}[1]{D{.}{\esperiod}{#1}}
\makeatletter
\addto\shorthandsspanish{\let\esperiod\es@period@code}
\makeatother

\usepackage{longtable}
\usepackage{tabu}
\usepackage{supertabular}

\usepackage{multicol}
\newsavebox\ltmcbox

% Para algoritmos
\usepackage{algorithm}
\usepackage{algorithmic}
\usepackage{amsthm}

% Para matrices
\usepackage{amsmath}

% Símbolos matemáticos
\usepackage{amssymb}
\usepackage{accents}
\let\oldemptyset\emptyset
\let\emptyset\varnothing

\usepackage[section]{placeins} % Para gráficas en su sección.
\usepackage[T1]{fontenc} % Required for accented characters
\newenvironment{allintypewriter}{\ttfamily}{\par}
\setlength{\parindent}{0pt}
\parskip=8pt
\linespread{1.05} % Change line spacing here, Palatino benefits from a slight increase by default

\makeatletter
\renewcommand\@biblabel[1]{\textbf{#1.}} % Change the square brackets for each bibliography item from '[1]' to '1.'
\renewcommand{\@listI}{\itemsep=0pt} % Reduce the space between items in the itemize and enumerate environments and the bibliography
\newcommand{\imagen}[2]{\begin{center} \includegraphics[width=90mm]{#1} \\#2 \end{center}}

\renewcommand{\maketitle}{ % Customize the title - do not edit title and author name here, see the TITLE block below
\begin{center} % Center align
{\Huge\@title} % Increase the font size of the title
\end{center}

\vspace{50pt} % Some vertical space between the title block and author block

\begin{flushright} %Right align
{\large\@author} % Author name
\\\@date % Date
\end{flushright}

\vspace{40pt} % Some vertical space between the author block and abstract
}

\usepackage[hidelinks]{hyperref}
%----------------------------------------------------------------------------------------
%	TITLE
%----------------------------------------------------------------------------------------

\title{\textbf{Fractales}\\ % Title
\vspace{10pt} % Some vertical space between title and subtitle
Matemática de belleza infinita} % Subtitle
%El subtítulo no se me ocurrió a mí, pero me encantó cuando lo leí.

\author{\textsc{Óscar Bermúdez} % Author
\\{\textit{Universidad de Granada}}} % Institution

\date{\today} % Date

%----------------------------------------------------------------------------------------

\begin{document}

\maketitle % Print the title section

{\parskip=2pt
  \tableofcontents
}
\pagebreak

\section{Definición}
	La definición más aceptada es:\\
	\textit{<<Un fractal es un objeto geométrico cuya estructura básica, fragmentada o irregular, se repite a diferentes escalas.>>}.
	
	El término fue propuesto por el matemático Benoît Mandelbrot(véase \ref{Homenaje}) y deriva del latín \textit{$"$fractus$"$}, que significa quebrado o fracturado, que hace referencia a la propiedad matemática clave de un fractal, que su dimensión métrica fractal es un número no entero. Aunque esta definición no es rigurosa.
	
	El concepto de fractal no dispone de una definición matemática precisa y de aceptación general. Hay varios intentos parciales:
	\begin{itemize}
		\item \textbf{B. Mandelbrot}, definió fractal como un conjunto cuya dimensión de Hausdorff-Besicovitch es estrictamente mayor que su dimensión topológica. Él mismo reconoció que su definición no era lo suficientemente general.
		\item \textbf{D. Sullivan}, que definió matemáticamente una de las categorías de fractales con su definición de conjunto cuasiautosimilar que hacía uso del concepto de cuasi-isometría.
	\end{itemize}
	
	Los fractales se caracterizan por:
	\begin{itemize}
		\item Ser demasiado irregulares como para ser descritos por la geometría tradicional.
		\item Ser autosimilar.
		\item Su dimensión Hausdorff-Besicovitch es estrictamente mayor que su dimensión topológica.
		\item Se define mediante un simple algoritmo recursivo.
	\end{itemize}
	
\section{Historia}
	
	\subsection{Fractales en la naturaleza}
	\subsection{Fractales básicos}
		%Imagen y pequeña descripción. Usar algunos como ejemplo de cálculo de dimensión fractal.
		\subsubsection{Triángulo de Sierpinski} \label{Sierpinski}
		\subsubsection{Alfombra de Sierpinski o esponja de Menger} \label{Menger}
		\subsubsection{Copo de nieve de Koch} \label{Koch}
		\subsubsection{Helecho de Barnsley} \label{Barnsley}
		\subsubsection{Escalera de Cantor o escalera del Diablo} \label{Cantor} % Sí, soy macabro, algún problema? XD
		\subsubsection{Curva de Lévy C o de Cesàro} \label{Cesàro}
		\subsubsection{Curva del Dragón} \label{Dragon}
		\subsubsection{Curva de Hilbert} \label{Hilbert}
		
	\subsection{Conjuntos de Julia} \label{Julia}
	\subsection{Conjunto de Mandelbrot} \label{Mandelbrot}
\section{Autosimilitud}
	Informalmente, la autosimilitud o autosemejanza, es la propiedad de un objeto en el que el todo es exacta o aproximadamente similar a una parte de sí mismo. Muchos objetos del mundo real son estadísticamente autosimilares como la romanescu. %Insertar imagen
	
	Matemáticamente, se dice que un conjunto compacto $K$ es autosimilar si existe un conjunto finito de homeomorfismos no sobreyectivos $\{H_1,\dots,H_n\}$ para el cual:
	\begin{equation*}
		K=\bigcup_{k=1}^n H_k(X).
	\end{equation*}
	
	Si $K \subset X$, decimos que $K$ es autosimilar si es el único subconjunto no vacío de $X$ tal que la ecuación anterior es válida para $\{ H_k \}_{k=1\dots n}$.
	
	Se pueden obtener diferentes tipos de similaridad según la naturaleza de los homeomorfismos $\{H_k\}_{k=1\dots n}$:
	
	\begin{itemize}
		\item Si son semejanzas exactas entonces el conjunto es autosimilar exacto.
		\item Si son aplicaciones afines entonces, el conjunto presentará autoafinidad.
		\item Si son aplicaciones conformes\footnote{En matemáticas, una aplicación es conforme si preserva ángulos.} entonces, el conjunto presentará autoconformidad.
	\end{itemize}
\section{Sistema de Funciones Iteradas(IFS)} \label{IFS}
\ref{Barnsley}
\section{Dimensión fractal}
\ref{Sierpinski} \ref{Cantor}
\section{Dimensión de Hausdorff-Besicovitch}
\ref{Hilbert}

%Algo por el 4º aniversario de su muerte el pasado 14 de noviembre. (Me lo tenía que haber preparado para el 18, fallo mío).
% Aunque se puede hacer algo para el 80º aniversario de su nacimiento(20 de noviembre)
\section{Homenaje a Mandelbrot} \label{Homenaje}
\href{https://www.youtube.com/watch?v=ES-yKOYaXq0}{Vídeo de Jonathan Coulton}\\
Su conjunto(\ref{Mandelbrot})\\
% En caso de no tener tiempo de prepararme la vida de Mandelbrot
\href{http://es.wikipedia.org/wiki/Beno\%C3\%AEt_Mandelbrot}{Para más información, véase Wikipedia}


\end{document}
